
% conclusion section for pedestrian study paper
% last updated : 11-dec-2018 23:45

\section{Conclusions and Future Work}

This paper presented an analysis to quantify the relative potential for pedestrian safety improvement for each of the 4,196 grid cells defined for this evaluation. Each of the grid cells represents an approximate 250 m by 250 m surface area within the city of Cincinnati, Ohio. The quantified PSI for each grid cell is determined based on the observational data that is acquired from publicly available sources. The data used to develop the expected cost severity functions are varied in nature – property values, requests for non-emergency service, access to public transportation, along with observed frequencies and severities of traffic accidents not involving pedestrians. This approach to pedestrian safety modeling that does not rely on specific infrastructure definitions (e.g., intersection types) or actual traffic counts (e.g., vehicle miles traveled per street segment) is a novel approach that the authors have not seen previously published.

The data sets utilized in this study are not in the range typically characterized as big data; however, the volume of data processed to arrive at the feature definitions is still substantial.  The extraction of predictive features from the non-emergency experiences required characterizing more than 600,000 individual service calls. The regression model indicates that a wide range of experiences from moderately-sized data sets can provide model predictive performance that compares favorably with traditional models that are based more closely on traffic patterns and infrastructure characteristics. This study produced r²-adj values of 0.34, while studies based on traffic counts and infrastructure definitions provide r²-adj values in the range of 0.25. Thus, the use of publicly available data that is recorded for reasons other than its use in this study provides a path to complete statistical studies of sufficient performance while at the same time reducing reliance on more costly data gathering methods (traffic counts, for instance).

The grid cells that are identified to have the greatest PSI are in the neighborhoods of :  Westwood, , Clifton, Avondale, Walnut Hills, and Bond Hill. These specific regions are recommended to be prioritized for remediation.

The unsupervised learning application completed here also identifies useful results to gain insights into the nature of unique advantages and disadvantages from the perspective of pedestrian interaction with their local environments. The results of the clustering analysis conducted from these data sets identify three unique clusters, each identifying different opportunities for improvement and each in a set of different neighborhoods. The results identify that crosswalks and visibility are problematic in the neighborhoods of  College Hill, Clifton, the Central Business District, and Over-the-Rhine; that speeding and lack of sidewalks are prevalent in  West Wood, West Price hill, Pleasant Ridge, and Kennedy Heights; and that vehicle violating traffic signals and failing to yield at crosswalks are dominant characteristics in  Sayler Park, East End, Mt. Washington, Winton Hill, and Carthage. It is worth noting that this classification of differing neighborhood characteristics was arrived at with the data sets which were not assembled for the purpose of understanding urban walkability needs, yet as modeled and evaluated from the unsupervised learning, useful insights are developed.

For future consideration, this work developed a solution using logistic and linear regression methods. The opportunity also exists to explore additional modeling methods from the machine learning toolkit to evaluate potential model improvements from that approach. Random forest modeling is a likely candidate next step, as random forest models can produce good accuracy in data sets of non-normal distributions. 

An obvious next step to evaluate improvements in this approach is to include the traditional vehicle miles traveled and infrastructure definitions as predictor variables. The combination of the features derived for this model along with the additional granularity of traffic patterns can likely provide additional improvements in model performance. This study demonstrates that data sets that characterize the interaction of citizens to their environment support reasonable model results; there is every reason to consider additional types of similar data (e.g., locations of schools, churches, hospitals, various types of establishments) as potential improvement opportunities. From a different aspect, the location tracking devices that are ubiquitous in cell phones and smart watches likely provide a level of detailed characterization between the movement of people and their environment, including development of time series interaction with local geography.



